Die Österreichischen Bundesbahnen (kurz ÖBB) sind ein Bahnunternehmen, welches österreichweit agiert. Das Technische Service Werk
in Linz beschäftigt sich mit leichter und schwerer Instandhaltung, Lackierung, Komponentenaufarbeitung, Unfallreperaturen,
Engineering-Leistungen, Umbauten beziehungsweise Modifikationen und der Überprüfung von Zugsicherungs- und Zugfunksystemen.
Diese Arbeit beschäftigt sich konkret mit der Komponentenaufarbeitung, genauer mit der Motorenaufarbeitung/Getriebeaufarbeitung.
Im TS-Werk Linz werden neue oder reparierte Antriebe, bevor sie in eine Lok eingebaut werden, gründlich auf Fehler getestet, jedoch
werden die auftretenden Fehler nur zur Laufzeit angezeigt und dann in unübersichtlichen Dateien abgespeichert. Das macht die 
spätere Fehlerauswertung und Prognose für zukünftige Antriebe schwer und umständlich. Hier kommt ASUQZ ins Spiel. ASUQZ bedeutet
Antriebsprüfstands-Software zur Untersuchung und Qualitätssicherung von Zugteilen und ermöglicht es, 
die Fehler nach verschiedenen Kriterien leicht und schnell abzurufen sowie graphisch darzustellen.